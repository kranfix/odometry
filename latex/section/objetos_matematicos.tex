\documentclass[main.tex]{subfiles}
\usepackage{amssymb}
\usepackage{mathtools}
\begin{document}

Los brazos robóticos ya son muy conocidos y utilizados pero el
valor agregado es la reconfiguracion/reprogramacion mediante un
brazo robótico a escala.

Indagar a nivel nacional la necesidad de las empresas, y presentarle
esta solución económica y reutilizable al ser de bajo costo y de alta
factibilidad.

Básicamente lo que se plantea es un brazo multifuncional, de esta manera
no hay necesidad de desechar un brazo, de esta manera disminuyendo el
impacto ambiental.


\subsection{Geometría 3D}
Para el desarrollo e implementación del código es necesario conocer
el modo de operación de las ecuaciones y funciones definidas. Por
tanto a continuación se muestran las definiciones y ecuaciones
previas al desarrollo del código.

\begin{itemize}
\item Puntos o vectores
  \begin{itemize}
  \item Vector
    \begin{equation}
      \mathbf{X} = \left(
      \begin{array}{ccc}x \\y \\z \\\end{array} \right)
      \epsilon \mathbb{R}^{3}
    \end{equation}
  \item Vector aumentado
    \begin{equation}
      \mathbf{\bar{X}} = \left(
          \begin{array}{ccc}x \\y \\z \\1 \\
          \end{array}        \right) \epsilon \mathbb{R}^{4}
    \end{equation}
  \item Coordenadas Homogéneas
    \begin{equation}
      \mathbf{\tilde{X}} = \left(
          \begin{array}{ccc}
            \tilde{x} \\ \tilde{y} \\ \tilde{z} \\ \tilde{w} \\
          \end{array}
        \right) \epsilon \mathbb{P}^{3}
    \end{equation}
  \end{itemize}
\item Rectas
  \begin{itemize}
  \item Recta sobre los puntos p y q
    \begin{equation}
      r = (1-\lambda )p+\lambda q
    \end{equation}
    \begin{equation}
      \lambda \epsilon \mathbb{R}
    \end{equation}
  \item Segmento de línea sobre p y q
    \begin{equation}
      r = (1-\lambda )p+\lambda q
    \end{equation}
    \begin{equation}
      0\leq \lambda  \leq 1
    \end{equation}
    \begin{figure}[h]
      \centering
      \includegraphics[width=0.2\textwidth]{../img/recta_3d.jpg}
      \caption{Recta 3D}
      \label{recta_3d}
    \end{figure}
  \end{itemize}
\item Planos:
  \begin{itemize}
  \item Representación vectorial del plano
    \begin{equation}
      \tilde{m}=\left ( a,b,c,d \right )^{T}
    \end{equation}
  \item Ecuación del plano
    \begin{equation}
      \tilde{m}=\left ( a,b,c,d \right )^{T}    
    \end{equation}
  \item Plano normalizado con vector unitario normal “n” y distancia “d”
    \begin{equation}
      m= \left( \hat{n_{x}},\hat{n_{y}},\hat{n_{z}},d \right) ^ {T}=
         \left( \hat{n},d \right)
    \end{equation}
    \begin{equation}
      \left \| \hat{n} \right \| = 1
    \end{equation}
    \begin{figure}[h]
      \centering
      \includegraphics[width=0.2\textwidth]{../img/plano_3d.jpg}
      \caption{Plano 3D}
      \label{plano_3d}
    \end{figure}
  \end{itemize}
\end{itemize}

\subsection{Transformaciones de coordenadas}
\begin{itemize}
\item Traslación
  \begin{equation}
    {\tilde{x}}'= \begin{pmatrix}
                    I & t \\
                    0^{T} & 1 \end{pmatrix} \tilde{x}
  \end{equation}
  
\item Rotación
  \begin{equation}
    {\tilde{x}}' = \begin{pmatrix}
                     R &1 \\
                     0^{T} &1
                   \end{pmatrix}\tilde{x}  
  \end{equation}
  \begin{equation}
    R = \begin{pmatrix}
          r_{11} & r_{12}  &r_{13} \\ 
          r_{21} & r_{22}  &r_{23} \\ 
          r_{31} & r_{23}  &r_{33}
        \end{pmatrix}
  \end{equation}

\item Traslación y rotación (Transformaciones Euclidianas)
  \begin{itemize}
  \item Traslación con 3 grados de libertad
    \begin{equation}
      t \epsilon \mathbb{R}^{3}
    \end{equation}
  \item Rotación con 3 grados de libertad
    \begin{equation}
      R\epsilon \mathbb{R}^{3x3}
    \end{equation}
    \begin{equation}
      X = \begin{pmatrix}
            R & t \\
            0 & 1 \end{pmatrix} \tilde{x}
    \end{equation}
    \begin{equation}
      X = \begin{pmatrix}
            r_{11} & r_{12}  & r_{13} & t_{1}\\ 
            r_{21} & r_{22}  & r_{23} & t_{2}\\ 
            r_{31} & r_{23}  & r_{33} & t_{3}\\
            0      & 0       & 0      & 1
          \end{pmatrix}
    \end{equation}
  \end{itemize}
\end{itemize}

\subsection{Ángulos de Euler (Roll-Pitch-Yaw)}
\begin{figure}[h]
  \centering
  \includegraphics[width=0.2\textwidth]{../img/rpy_drone.jpg}
  \caption{Drone con sus ejes de Euler}
  \label{rpy_drone}
\end{figure}

\subsection{Axis-Angle (eje-ángulo)}

\begin{figure}[h]
  \centering
  \includegraphics[width=0.2\textwidth]{../img/axisang.jpg}
  \caption{Ejemplo de Axis-angle}
  \label{axisang}
\end{figure}

Conversiones (Fórmula de Rodriguez)
\begin{equation}
  R(\hat{n},\theta)=I+\sin\theta\left [ \hat{n} \right ]_{x}+(1-\cos\theta)\left [ \hat{n} \right ]_{x}^2
\end{equation}
\begin{equation}
  \theta=\\cos^{-1}(\frac{trace(R)-1}{2})
\end{equation}
\begin{equation}
  \hat{n} = \frac{1}{2\sin\theta}
            \begin{bmatrix}
              r_{32}-r_{23}\\ 
              r_{13}-r_{31}\\ 
              r_{21}-r_{12}
            \end{bmatrix}
\end{equation}

\subsection{Quaterniones}
\begin{itemize}
\item Representación vectorial
  \begin{equation}
    q=\left ( q_{w},q_{x},q_{y},q_{z} \right )^T \epsilon \mathbb{R}^4
  \end{equation}
  \begin{equation}
    q=\left ( r,v \right )^T  
  \end{equation}
\item Quaternion unitario
  \begin{equation}
    \left \| q \right \|=1
  \end{equation}
\item Relación Axis-angle
  \begin{equation}
    q = \left( r,v \right) ^ T
      = \left(
          \cos\frac{\theta}{2},
          \sin\frac{\theta}{2} \hat{n}
        \right) ^ T
  \end{equation}
\end{itemize}

\end{document}
