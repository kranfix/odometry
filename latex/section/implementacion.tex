\documentclass[main.tex]{subfiles}
\begin{document}

La forma de implementar los objetos matemáticos es mediante la creación de una biblioteca que determine operaciones vectoriales y otras que se encarguen de definir y transformar diferentes sistemas de referencia. Por tanto a continuación procedemos a definir a estas:

\begin{itemize}
\item vector:
	permite operaciones vectoriales hasta de orden 3.
\item mat3: 
	representación matricial de las Transformaciones Eucledianas, permitiendo el manejo de 			operaciones entre matrices cuadradas de 3x3.
\item rpy:
	representación de los Ángulos de Euler.
\item axisang:
	representación del ángulo normal y vector axial.
\item quaternion:
	representación de quaterniones unitarios (parámetros de Euler-Rodrigues) permitiendo otro 		mecanismo de representar rotaciones 3D.
\end{itemize}

\end{document}
